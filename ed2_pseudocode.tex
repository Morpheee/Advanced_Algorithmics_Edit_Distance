\documentclass[12pt,a4paper]{article}

\usepackage[utf8]{inputenc}
\usepackage[T1]{fontenc}
\usepackage[english]{babel}
\usepackage{lmodern}
\usepackage{enumerate}
\usepackage{marvosym}
\usepackage{tabularx}
\usepackage{amsmath}
\usepackage{amsfonts}
\usepackage{amssymb}
\usepackage{mathtools}
\usepackage{graphicx}
\usepackage{fancyhdr}%entt/pdp
\usepackage{amsthm}
\usepackage{mathrsfs}
\usepackage{titlesec}
\usepackage{hyperref}
\usepackage{tikz}
\usepackage{enumerate}
\usepackage{cancel}
\usepackage{multicol}
\usepackage{mdframed}
%\usepackage{bbold}
\usepackage[left=1.00cm, right=1.00cm, top=3.00cm, bottom=2.00cm]{geometry}
\usepackage{chngcntr}
\usepackage{xcolor}
\usepackage{array}
\usepackage{stmaryrd}
\usepackage{siunitx}



\hypersetup{
	colorlinks=true, %set true if you want colored links
	linkcolor=black,
}

% ********** NEWCOMMAND **********
\newcommand{\p}{\bullet}
\newcommand{\tab}{\hspace*{0.5cm}}
\newcommand{\tabb}[1][\hspace*{1cm}]{\hspace*{#1 cm}}
\newcommand{\vtab}{\vspace*{0.2cm}}
\newcommand{\vtabb}[1][\vspace*{0.2cm}]{\vspace*{#1cm}}


% ********** ENTETE/PIED DE PAGE **********
\pagestyle{fancy}
\fancyhf{}
\rhead{Pseudocode}
\lhead{Advanced Algorithm}
\rfoot{Page \thepage}

% ********** ALGORITHMS **********
\usepackage[linesnumbered,ruled,vlined]{algorithm2e}
\newcommand\mycommfont[1]{\footnotesize\ttfamily\textcolor{blue}{#1}}
\SetCommentSty{mycommfont}
\SetKwInput{KwInput}{Input}                % Set the Input
\SetKwInput{KwOutput}{Output}              % set the Output
\renewcommand{\l}{$\ell$}
\newcommand{\la}{$ \leftarrow $}

\begin{document}
	\begin{algorithm}[H]
		\KwInput{X, Y}
\tcc{
Parameters :\\
\tab X , Y : strings\\
Return :\\
\tab ed : integer, optimal edit distance between X and Y.\\
\tab alignment : array of instructions, to go from Y to X.\\
}
	R \la ed\_dynamic\_mat(X,Y)\tcp{top-left to bottom right}
	L \la backward\_ed\_dynamic\_mat(X,Y)\tcp{bottom right to top-left}
	S \la R+L\;
	ed \la S[0,0]\;
	alignment \la extract optimal "path" from S.\;
\tcc{
The optimal path is one of the possible path which goes from top-left to bottom-rigth cell of S, using only the cells with minimum number (the edit distance).
}
	\Return \{ed : ed , alignment : alignment\}
	\caption{Dynamic\_Programming\_Approach\_Edit\_Distance\_bis}
	\end{algorithm}	
\end{document}