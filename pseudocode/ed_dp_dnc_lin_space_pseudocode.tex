\documentclass[12pt,a4paper]{article}

\usepackage[utf8]{inputenc}
\usepackage[T1]{fontenc}
\usepackage[english]{babel}
\usepackage{lmodern}
\usepackage{enumerate}
\usepackage{marvosym}
\usepackage{tabularx}
\usepackage{amsmath}
\usepackage{amsfonts}
\usepackage{amssymb}
\usepackage{mathtools}
\usepackage{graphicx}
\usepackage{fancyhdr}%entt/pdp
\usepackage{amsthm}
\usepackage{mathrsfs}
\usepackage{titlesec}
\usepackage{hyperref}
\usepackage{tikz}
\usepackage{enumerate}
\usepackage{cancel}
\usepackage{multicol}
\usepackage{mdframed}
%\usepackage{bbold}
\usepackage[left=1.00cm, right=1.00cm, top=3.00cm, bottom=2.00cm]{geometry}
\usepackage{chngcntr}
\usepackage{xcolor}
\usepackage{array}
\usepackage{stmaryrd}
\usepackage{siunitx}



\hypersetup{
	colorlinks=true, %set true if you want colored links
	linkcolor=black,
}

% ********** NEWCOMMAND **********
\newcommand{\p}{\bullet}
\newcommand{\tab}{\hspace*{0.5cm}}
\newcommand{\tabb}[1][\hspace*{1cm}]{\hspace*{#1 cm}}
\newcommand{\vtab}{\vspace*{0.2cm}}
\newcommand{\vtabb}[1][\vspace*{0.2cm}]{\vspace*{#1cm}}


% ********** ENTETE/PIED DE PAGE **********
\pagestyle{fancy}
\fancyhf{}
\rhead{Pseudocode}
\lhead{Advanced Algorithm}
\rfoot{Page \thepage}

% ********** ALGORITHMS **********
\usepackage[linesnumbered,ruled,vlined]{algorithm2e}
\newcommand\mycommfont[1]{\footnotesize\ttfamily\textcolor{blue}{#1}}
\SetCommentSty{mycommfont}
\SetKwInput{KwInput}{Input}                % Set the Input
\SetKwInput{KwOutput}{Output}              % set the Output
\renewcommand{\l}{$\ell$}
\newcommand{\la}{$ \leftarrow $}

\begin{document}
	\begin{algorithm}[H]
		%\DontPrintSemicolon

		\KwInput{X , Y}
\tcc{
Parameter :\\
	\tab x,y : strings.\\
Return :\\
	\tab l[1,:] : the last line of the array created by the dynamic programming approach.
}
		\Begin{
\tcc{Blank letter added to X and Y corresponds to the uppest row and leftest column of the array \l.}
			X \la concat($\emptyset$,X)\;
			Y \la concat($\emptyset$,X)\;
			n \la len(X)\;
			m \la len(Y)\;
			Create Array \l~of size [0..1 , 0..n]\;
			Initialize \l~:\\
			{
\tcc{Initializing \l~with uppest row equal to 0,1,...,n-1 and leftest column equal to 0,1,...,m-1. As \l~is composed of two rows (for space efficiency), the edge case corresponding to the leftest column must be handelded in the main loop.}
				\tab\l[0 , i] = i for i$\in$[0..n]\;
				\tab\l[1 , 0] = 1\;
			}
\tcc{Algorithm's main loop.}
			\For{i \la 1 to m}
			{
				\For{j \la 1 to n}
				{
\tcc{Edge case : the i-nth leftest element must be equal to i}
					\If{X[j] == Y[i]}
					{
						\l[1 , j] \la \l[0 , j-1]\;
					}
					\Else
					{
						\l[1 ,j] \la 1 + $\min$ \begin{tabular}{|l}
										\l[0 , j]\\
										\l[0 , j-1]\\
										\l[1 , j-1]
						\end{tabular}
					}
				}
\tcc{Make uppest row equal to the lowest before starting again.}
				\l[0 , :] \la \l[1 , :]
			}
			\Return \l[1 , :]
		}
		\caption{Space\_Efficient\_Dynamic\_Programming\_Edit\_Distance}
	\end{algorithm}
	\newpage	
	\begin{algorithm}[H]
		%\DontPrintSemicolon
		
		\KwInput{X , Y}
\tcc{
Parameter :\\
	\tab x,y : strings.\\
Return :\\
	\tab r[0,:] : the first line of the array created by the backward dynamic programming approach.
}
		\Begin{
			\tcc{Blank letter added to X and Y corresponds to the lowest row and rightest column of the array r.}
			X \la concat($\emptyset$,X)\;
			Y \la concat($\emptyset$,X)\;
			n \la len(X)\;
			m \la len(Y)\;
			Create Array r~of size [0..1 , 0..n]\;
			Initialize r~:\\
			{
				\tcc{Initializing r~with lowest row equal to 0,1,...,n-1 and rightest column equal to 0,1,...,m-1. As r~is composed of two rows (for space efficiency), the edge case corresponding to the rightest column must be handelded in the main loop.}
				\tab r[1 , i] = n-i for i$\in$[1..n+1]\;
				\tab r[0 , n-1] = 1\;
			}
			\tcc{Algorithm's main loop.}
			\For{i \la m-2 to -1 step -1}
			{
				\For{j \la n-1 to -1 step -1}
				{
					\tcc{Edge case : the (m-i)-nth rightest element must be equal to (m-i-1)}
					\If{X[j] == Y[i]}
					{
						r[0 , j] \la r[1 , j+1]\;
					}
					\Else
					{
						r[1 ,j] \la 1 + $\min$ \begin{tabular}{|l}
							r[1 , j]\\
							r[1 , j+1]\\
							r[0 , j+1]
						\end{tabular}
					}
				}
\tcc{Make lowest row equal to the uppest before starting again.}
				r[1 , :] \la r[0 , :]
			}
			\Return r[0 , :]
		}
		\caption{Backward\_Space\_Efficient\_Dynamic\_Programming\_Edit\_Distance}
	\end{algorithm}
	\newpage
	\begin{algorithm}
		\KwInput{X , Y}
\tcc{
Parameter :\\
	\tab x,y : strings, the two words we work with.\\
Return :\\
	\tab \{'p' : p, 'ed' : ed\} dictionnary :\\
	\tab$\p$ p : the collection of points of interest to compute the alignement between x and y.\\
	\tab$\p$ ed : the edit distance between x and y.
}
		\Begin{
			n \la len(X)\;
			m \la len(Y)\;
			\If{n < 1 or m < 2}{
\tcc{Edge cases, if (len(X) == 0,1) or (len(Y) == 0,1)}
				\If{m == 1 and n > 1}{
					l \la Space\_Efficient\_Dynamic\_Programming\_Edit\_Distance(X,Y)\;
					s \la l + [n,n-1,...,0]\;
					mini \la leftest index minimizing s\;
					p = [(mini , m)]\;
				}
				\ElseIf{n == 1 and m > 0}{
\tcc{if X is in Y then there are m-1 operations to do to go from Y to X, else there are m changes to do.}
					ed = m-1 \textbf{if} x in y \textbf{else} m\;
					p = [(X , Y[0])]\;
				}
				\Else{
					Handle case X empty or Y empty\;			
				}
				\Return \{ 'p' : p , 'ed' : ed \}\;
			}
			\Else{
				\l~\la Space\_Efficient\_Dynamic\_Programming\_Edit\_Distance(X,Y[:$ \lfloor\frac{\text{m}}{2}\rfloor $])\;
				r \la  Backward\_Space\_Efficient\_Dynamic\_Programming\_Edit\_Distance(X,Y[$ \lfloor\frac{\text{m}}{2}\rfloor $:])\;
				s \la \l + r\;
				mini \la leftest index minimizing s\;
				ed = s[mini]\;
				p = [(mini , $ \lfloor\dfrac{\text{m}}{2}\rfloor $)]\;~\\
				\tab \textbf{Concatenate}(p , Divide\_and\_Conquier\_Edit\_Distance(X[:mini] , Y[:$ \lfloor\frac{\text{m}}{2}\rfloor $])['p']) \tcp*{top-left}
				\tab \textbf{Concatenate}(p , Divide\_and\_Conquier\_Edit\_Distance(X[mini:] , Y[$ \lfloor\frac{\text{m}}{2}\rfloor $:])['p']) \tcp*{bottom-right}
\tcc{\textbf{concatenate} : w.r.t. original X and Y coordinates and such that the final p is still sorted$ ^\ast $.}
				\Return \{ 'p' : p , 'ed' : ed \}\;
\tcc{Post-treat p to get alignement from the points of interest.}
\tcc{$ \ast $ : here }
			}
		}
		\caption{Divide\_and\_Conquier\_Edit\_Distance}
	\end{algorithm}

	
\end{document}